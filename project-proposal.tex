%        File: project-proposal.tex
%     Created: Wed Oct 11 11:00 PM 2017 E
% Last Change: Wed Oct 11 11:00 PM 2017 E
%
\documentclass[10pt]{article}

\usepackage[margin=1in]{geometry}
\usepackage{fancyhdr}
\usepackage{tikz}

\pagestyle{fancy}

\lhead{CS 281}
\rhead{Final Project Proposal}

\begin{document}

\title{Final Project Proposal}
\author{
Kevin Chen              \\ \texttt{jiafengchen@college.harvard.edu}
\and Yufeng Ling        \\ \texttt{yufengling@college.harvard.edu}
\and Francisco Rivera   \\ \texttt{frivera@college.harvard.edu}}
\date{\today}

\maketitle

\section{Problem Description}

The primary objective of this project is to predict trip duration from taxi data
published by the TLC in New York City. An accurate predictor has a host of
implications, among them,

\begin{itemize}
\item Accurate predictions can allow NYC residents to give themselves a smaller
time buffer when commuting in the city, and thus save time.

\item Estimates of trip duration play a central role in the pricing of trips by
ride-sharing apps, which increasingly feel pressured to a quote a price before
the completion of the trip.

\item From the perspective of a city planner, easing congestion requires an
understanding of which trips systematically take longer.
\end{itemize}

\section{The Data}

\emph{Talk about the data that we get here. Columns we observe, time period we
get, and how much data there is.}


\section{Methodology}

\subsection{Latent Variables Model}

\begin{center}
\begin{tikzpicture}
\draw[->] (0,0) -- (0,5);
\draw[<-] (2,0) -- (2,5);
\draw[->] (4,0) -- (4,5);
\draw[->] (0,0) -- (5,0);
\draw[<-] (0,2) -- (5,2);
\draw[->] (0,4) -- (5,4);

\node[below left] at (0,0) {$(0,0)$};
\node[left] at (0,1) {$\lambda_{0,0}^{\uparrow}$};
\node[left] at (0,1) {$\lambda_{0,0}^{\uparrow}$};
\node[below] at (1,0) {$\lambda_{0,0}^{\rightarrow}$};
\end{tikzpicture}
\end{center}



\end{document}


